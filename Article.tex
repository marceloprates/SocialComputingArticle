%%%% ijcai11.tex

\typeout{IJCAI-13 Instructions for Authors}

% These are the instructions for authors for IJCAI-13.
% They are the same as the ones for IJCAI-11 with superficical wording
%   changes only.

\documentclass{article}
% The file ijcai13.sty is the style file for IJCAI-13 (same as ijcai07.sty).
\usepackage{ijcai13}

% Use the postscript times font!
\usepackage{times}

\usepackage{latexsym} 


\title{Social Computing Article\thanks{This title is temporary}}
\author{Tabajara, Lucas \\
Institute of Informatics \\
Federal University of Rio Grande do Sul\\
Brazil \\
lmtabajara@inf.ufrgs.br
\and
Prates, Marcelo \\
Institute of Informatics \\
Federal University of Rio Grande do Sul\\
Brazil \\
morprates@inf.ufrgs.br}

\begin{document}

\maketitle

\begin{abstract}
  Abstract
\end{abstract}

\section{Introduction}

For years multi-agent systems have been used to research cooperation as a tool for problem-solving. Recently, however, there has been an increasing interest in the study of human beings as problem-solving agents. Several experiments have been conducted in which subjects are connected in a network with the goal of collectively solving a specific problem, and those have helped shed some light on the way humans interact to solve problems. However, although those studies can provide us with observations and hypotheses, there is still a difficulty in finding ways to explain the observed behaviour. In this paper, we use a multi-agent based simulation to complement the study of human computation, as a way of explaining the strategies used by humans and understanding their consequences in a cooperative environment. The control that simulations provide us over the behaviour of the agents allows us to better understand possible reasons for the ones displayed by humans.

Human beings are known to be able to easily perform tasks which are still generally difficult for computers, such as natural language processing and image recognition. However, it would be useful to be able to apply human social computation to more straightforward computer problems with precise definitions and algorithms, but which are still computationally intensive. The different sets of abilities between computers and humans suggest that the latter might provide a new approach to those problems that might be more effective than current techniques employed by computers. That notion has been successfully applied, for example, in one of the most significant problems in the field of bioinformatics: protein structure prediction (PSP). A software that presents the problem to humans as an online computer game, called \emph{Foldit} \cite{cooper:foldit}, has produced significant results in the research of PSP, which is usually approached as an optimization problem requiring extensive computational power. Those results have been attributed to human visual problem-solving and decision-making abilities, as well as social collaboration \cite{cooper:foldit}. However, we still don't know the limits of human abilities in problem-solving and how they compare to more traditional techniques. To take full advantage of human problem-solving strategies, we must learn their limitations. In order to obtain that knowledge, our system simulates human behaviour according to findings of social computation experiments.

\section{Background}

Social computation is a relatively new area of research with a diversified, interdisciplinary root that mixes social sciences, artificial intelligence, game theory and network science, among others. Only recently studies on the potential of human social networks for solving problems are gaining some popularity. Those have provided insights into, among other things, the impact of network structure in the collaboration process and the factors that lead neighbours' proposed solutions to be copied by individuals.

The origin of human computation as we know it today can be traced back to the work of \cite{vonahm:gwap}, which identified the possibility of using entertainment as an incentive to participation of human subjects, applying it in games in which the participants are actually performing a computation. That is an idea that also appears in the Foldit game.

The series of experiments summarized in \cite{kearns:experim} are among the first to try to take advantage of the collective's problem-solving abilities to solve classical computer problems. The experiments are mostly based on the concept of coordination: subjects have individual incentives that are expected to drive them to cooperate with one another and lead them toward the collective goal.

Other initiatives have appeared, such as the ones by \cite{farenzena:collabem} and \cite{mason:collablearnet}, which take a different approach by having subjects trying to solve the collective problem individually, with the possibility of exchanging solutions between neighbours. Those have resulted in interesting conclusions on human behaviour when the possibility of copying peers makes itself present. It's in those models of experiments that we will focus in this work.

The experiments conducted by \cite{farenzena:collabem} had human beings trying to solve constraint satisfaction problems, namely Boolean Satisfiability (SAT) and the popular \emph{Sudoku} game, with individuals connected through the network being able to exchange partial solutions of the problem in question. There were two main patterns of behaviour observed in this study in regard to copying neighbours' solutions:

\begin{itemize}
	\item Humans might not evaluate the solutions proposed by their peers, instead choosing the most readily available one. On the other hand, \cite{mason:collablearnet} suggest that their subjects did evaluate the available solutions, although an in-depth analysis of that wasn't provided.
	\item There is a higher probability of individuals copying peers' solutions when those are shared by several neighbours. That behaviour is referred to by the authors as \emph{conformism}. A similar behaviour was observed by \cite{mason:collablearnet}, even though the experiments used a different problem.
\end{itemize}

\cite{mason:collablearnet} also provided important insights to human cooperation. However, since our basis is the work of \cite{farenzena:collabem} we will use the former's results only as secondary support.

\section{Contribution}

We have built a multi-agent system that simulates the experiments of \cite{farenzena:collabem}, using the model of human behaviour proposed by them. We adapted that behaviour into a set of rules which we modeled in the \emph{Memetic Network} model proposed by \cite{lamb:memenet}. For simplicity, we limited our experiments to the Sudoku problem. Our goal is verifying whether the strategies employed by humans are competitive with other heuristics. After identifying the points in which those are lacking, we will then propose methods to increase the efficiency of human problem-solving networks.

\section{System Overview}

We developed a system in which the environment of the {\em Sudoku} problem-solving social network presented in~\cite{farenzena:collabem} could be modelled through autonomous agent networks and analysed through a series of experiments. The optimization technique of {\em Memetic Networks}~\cite{lamb:memenet} was employed on the modelling, providing a basis for the dynamics of the agent network.

\section{Sudoku Solving Memetic Network}
 
A {\em Memetic Network Algorithm} is composed of an ordered set of $N$ agents, each encoding a complete solution to the optimization problem, and a binary $N \times N$ matrix representing possible connections between agents. Additionally, a {\em Memetic Network Algorithm} is composed of a set of rules specifying how connections between agents are formed and erased and how interactions between these agents take place. These rules are grouped under the categories of {\em Connection Rules}, specifying how agents will connect and disconnect from each other; {\em Aggregation Rules}, specifying the dynamics of the information flow through connections; and {\em Appropriation Rules}, specifying how agents are supposed to add local changes to the information received through their connections~\cite{lamb:memenet}.

Our solution adapts the {\em Memetic Newtork} technique - originally intended for use in optimization problems - to the context of constraint satisfaction problems. In this specific case ({\em Sudoku} solving), we treat problems with one unique solution. To deal with this scenario, we propose a {\em Memetic Network} variation in which each agent encodes not a complete, but a partial solution to the problem.

Our solution employs a fixed topology for the agent network and models cooperation through a set of {\em aggregation rules} and reasoning through a set of {\em appropriation rules}. Our aggregation rules specify how {\em Sudoku} partial solutions are copied from agent to agent, while our appropriation rules specify how agents increment these solutions with {\em Sudoku} solving techniques such as those studied in~\cite{davis:mathsudoku}.

\subsection{Aggregation Rules}

The experiments conducted by~\cite{farenzena:collabem} point out a series of observations about the dynamics of cooperation in problem solving with human beings. For instance, the authors analysis has shown that users tend to copy not the most complete solutions, but, instead, the first solutions on the graphic interface. Additionally, we know from this same paper that users rarely copy solutions from more than 6 neighbours. In order to analyse and compare this behaviours, we developed a collection of aggregation rules, each modelling a copying strategy.

\subsubsection{Pick Most Filled}

This rule employs the intuitive strategy of copying from the neighbour with the most filled {\em Sudoku} board, namely the most complete solution.

\subsubsection{Pick Among First}

We know that users tend to copy the first (from left to right) solutions on the graphic interface~\cite{farenzena:collabem}. The authors have provided us with a mathematical model of this behaviour, stated as $\langle X(k)\rangle = (1-p)^{k-1}p$, where the parameter $p$ is fixed as $p = 0.5479$ and $\langle X(k)\rangle$ denotes the probability of an agent copying the $k_{th}$ neighbour solution.

We inserted this behaviour into our model by firstly generating a random ordering of neighbours for each agent in order to compose a simulated graphic interface. As a result, each agent visualizes some neighbours before or after others. Secondly, we translated the above mathematical model into an aggregation rule in which the solution copied by an agent is the $k_{th}$ with probability $X(k)$.

\subsection{Appropriation Rules}

In {\em Sudoku} strategy literature, as in chess, we find multiple techniques with iconic names. Some popular examples are the {\em naked singles} rule, the {\em hidden singles} rule and the {\em naked twins} rule, studied in ~\cite{davis:mathsudoku}. These strategies intend to, given a {\em Sudoku} puzzle in a partial state of completion, generate movements to mark blank cells of the puzzle, as {\em "mark cell 1 of column 3 with value 5"}. We reproduced a variety of these strategies, modelling each one of them as a function that maps a {\em Sudoku} partial solution to a set of movements.

\subsubsection{Only Choice}

This technique marks a {\em Sudoku} cell with a value only if this cell is the last blank cell on its row, column or $3 \times 3$ block.

\subsubsection{Naked Singles}

This rule marks a {\em Sudoku} cell with value $v$ only if $v$ is the single possible value to mark that cell. The other values must have been eliminated through a process of verifying that they are all present in the span encompassing the cell's row, column and $3 \times 3$ block.

\subsubsection{Hidden Singles}

This rule is similar to the {\em naked singles} rule. It is able to mark cells with more than one candidate value by looking for cells that are the only cells in their row, column or $3 \times 3$ block that can hold a particular value.

\subsubsection{Two out of Three}

This rule applies to groups of three contiguous $3 \times 3$ blocks. It aims to find a value $v$ such that $v$ is present in two of the three $3 \times 3$ blocks encompassed by the group, but missing on the third. It proceeds by enumerating all the candidate cells - namely the empty cells - on this block, and then, by eliminating from this set all the cells that are encompassed by the rows or columns in which $v$ is placed on the other two blocks. If the resulting set has cardinality $1$, the rule has successfully found one cell to mark.

\subsubsection{Naked Twins}

\section{Results}

Results

\section{Conclusions}

Conclusions

%% The file named.bst is a bibliography style file for BibTeX 0.99c
\bibliographystyle{named}
\bibliography{Article}

\end{document}

